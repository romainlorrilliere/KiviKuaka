\documentclass[10pt,compress]{beamer}
\usepackage[utf8]{inputenc}
%\usepackage[T1]{fontenc}
\usepackage[english]{babel}
% \usepackage{natbib}
%\usepackage{hyperref}
% \usepackage{animate}
%\usepackage{graphicx}
%\usepackage{color}

\pdfminorversion=5 
\pdfcompresslevel=9
\pdfobjcompresslevel=3

%% Theme
% \useinnertheme[shadow=true]{rounded}
% \useoutertheme{shadow}
% \usecolortheme{orchid}
% \usecolortheme{whale}%\usetheme{progressbar}
%\usetheme[]{Berlin}

\useoutertheme[footline=authortitle,subsection=false]{miniframes}
%\useoutertheme{smoothtree}
\useinnertheme[shadow=true]{rounded}
\usecolortheme{orchid}
\usecolortheme{whale}

\usepackage{etoolbox}
\makeatletter
\patchcmd{\slideentry}{\advance\beamer@tempdim by -.05cm}{\advance\beamer@tempdim by\beamer@vboxoffset\advance\beamer@tempdim by\beamer@boxsize\advance\beamer@tempdim by 1.2\pgflinewidth}{}{}
\patchcmd{\slideentry}{\kern\beamer@tempdim}{\advance\beamer@tempdim by 2pt\advance\beamer@tempdim by\wd\beamer@sectionbox\kern\beamer@tempdim}{}{}
\makeatother

\graphicspath{{./image/}} 

% \usecolortheme{progressbar}
 %\usefonttheme{progressbar}
 %\useinnertheme{progressbar}
 %\useoutertheme{progressbar}
%% Logo
% \logo{\includegraphics[height=0.5cm]{./images/Logo.png}}
%% Fond de la page
 %\setbeamercolor{background canvas}{bg=couleur}

\title[Wandering tattlers of Tikei]{Overlapping territories in a small wintering population of wandering tattler \\in French Polynesia}  

\author[Lorrillière \& Jiguet]{Romain Lorrillière
  \footnotesize{\textit{romain.lorrilliere@mnhn.fr}} \&  Frédéric Jiguet}
\institute{MNHN - CESCO}
\date{SFE$^2$-GfÖ-EEF Metz 2022-11-23}

\titlegraphic{
\includegraphics[width=\textwidth]{title-tattler}
}

%\AtBeginSection[]{
%  \begin{frame}
%   \frametitle{\insertsectionhead}
%    \scriptsize \tableofcontents[currentsection,hideothersubsections]
%  \end{frame}
%}



%\AtBeginSubsection[]{
%  \begin{frame}
%    \frametitle{\insertsubsectionhead}
%    \scriptsize \tableofcontents[sectionstyle=show/shaded,subsectionstyle=show/shaded/hide ]
%  \end{frame}
%}
%\addtobeamertemplate{footline}{\insertframenumber/\inserttotalframenumber}

\begin{document}
\maketitle



% \section{Sommaire}
%% =====================
%\begin{frame}
%   \frametitle{Sommaire}
%   \tableofcontents[hideallsubsections]
% \end{frame}




% ===============================================
\section{Introduction}
% ===============================================


\begin{frame}
  \frametitle{The shorebirds}
  \begin{columns}
    \begin{column}[c]{0.5\textwidth}
      \begin{center}
        \includegraphics<1->[width=\textwidth]{shorebird_murmuring_2}
      \end{center}
    \end{column}
    \begin{column}[c]{0.5\textwidth}
      \begin{center}
        \includegraphics<2>[width=\textwidth]{shorebird_foraging_2}     
      \end{center}
    \end{column}
  \end{columns}
\end{frame}

\begin{frame}{The shorebirds: a threatened taxon}
 \begin{columns}
    \begin{column}[c]{0.55\textwidth}
      \begin{itemize}[<+->]
      \item about $40 \%$ decline in species {\tiny \cite{Zoeckler2003,Studds2017}}
      \item many threats {\tiny \cite{Sutherland2012}}
        \begin{itemize}
        \item human threats \\{\footnotesize ex: climatic change,
            degradation and destruction of stopover shore habitat}
        \item next threats \\{\footnotesize ex:agricultural intensification, pollution and disease, eutrophication}
        \item natural threats \\{\footnotesize ex: tsunami,
            storm, volcanism, earthquake}
        \end{itemize}
      \end{itemize}
    \end{column}
    \begin{column}[c]{0.45\textwidth}
      \begin{center}
       \includegraphics[width=\textwidth]{limi}     
      \end{center}
    \end{column}
  \end{columns}
\end{frame}


\begin{frame}{Stopover and wintering are crucial times for conservation}
  \begin{columns}
    \begin{column}[c]{0.60\textwidth}
      \begin{itemize}[<+->]
      \item The shorebird are long-distance migratory species
      \item The nutritional quality of stopover sites {\tiny
          \cite{Morrison2007,Studds2017}} and wintering areas {\tiny
          \cite{Piersma1993,Tulp2009}}  are major drivers to productivity
      \item In tropical area the avalaibles ressources could be limited
      \item These limitations drive competition that is the major
        driver to the shorebird survival during the winter {\tiny \cite{Baker1973}}
      \end{itemize}
    \end{column}
    \begin{column}[c]{0.4\textwidth}
      \begin{center}
        \includegraphics<1>[width=\textwidth]{migration_godwit}
        \includegraphics<2>[width=\textwidth]{feeding}
        \includegraphics<3->[width=\textwidth]{tropic}
      \end{center}
    \end{column}
  \end{columns}
\end{frame}


\begin{frame}{Interactions}
  \begin{columns}
    \begin{column}[c]{0.6\textwidth}
      \begin{itemize}[<+->]
      \item Territorial behaviours on the wintering areas
      \item Especially for species that visual feeders detected prey over large areas\\ {\footnotesize
          plovers, sandpipers and tattlers} {\tiny \cite{Colwell2000}}
      \end{itemize}
    \end{column}
    \begin{column}[c]{0.4\textwidth}
      \includegraphics<1>[width=\textwidth]{fight}
      \includegraphics<2>[width=0.7\textwidth]{plover-sandpiper-tattler}     
    \end{column}
  \end{columns}
\end{frame}


\begin{frame}{Many studies about wintering on mudflat and wetland }
  \begin{columns}
    \begin{column}[c]{0.55\textwidth}
      \begin{itemize}[<+->]
      \item Many studies about wintering on mudflat and wetland \\
        {\footnotesize Europe, Asia, South America, Australia,
          New-Zeland}
        \begin{itemize}
        \item tidal effects on habitat using
        \item polution effects on feed productivity
        \item interaction effects between many species in one place
        \end{itemize}
      \end{itemize}
    \end{column}
    \begin{column}[c]{0.45\textwidth}
      \includegraphics[width=\textwidth]{KiviKuaka_3_RL_20140912_071004_P9120201}
      \end{column}
  \end{columns}
\end{frame}

\begin{frame}{Many studies about wintering on mudflat and wetland }
  \begin{columns}
    \begin{column}[c]{0.55\textwidth}
      \begin{itemize}[<+->]
      \item In the Pacific Ocean, despite the absence of these productive habitats on the many reef islands, there are wintering birds
      \item For conservation issue it seems to be important to
        understand how the birds interact on these reef island for food
       \end{itemize}
    \end{column}
    \begin{column}[c]{0.45\textwidth}
      \includegraphics[width=\textwidth]{KiviKuaka_3_RL_20210127_124920_P1270001}
     \end{column}
  \end{columns}
\end{frame}



\begin{frame}{Our main questions}
  What is the spatial structure of a territorial shorebird population on a reef island?  
  \begin{itemize}
  \item Probably important overlap of bird home ranges 
  \item But despite the overlap, the birds should stay away from each other..
 \end{itemize}
\end{frame}




% ===============================================
\section{Material}
% ===============================================


 
\begin{frame}{Tikei: a tiny isolated island}
  \begin{center}
     \includegraphics<1-2>[width=0.6\textwidth]{polynesie_sites_1}   
     \includegraphics<3-4>[width=0.6\textwidth]{Tikei_2}
     \includegraphics<5-6>[width=0.6\textwidth]{KiviKuaka_3_RL_20210128_032720_DJI_0559-1} 
  \end{center}
  \begin{itemize}[<+->]
  \item Little island of Tuamotu archipelago in French Polynesia    
  \item $70 km$ to the closest islands
  \item $3.9 km$ in lenght and $1.6 km$ in width
  \item $4.5 km^2$  
  \item altitude max $3 m$
  \item $9 km$ de côte
  \end{itemize}
\end{frame}

\begin{frame}{Tikei: a tiny isolated island}
  \begin{center}
    \includegraphics<1-4>[width=0.6\textwidth]{Tikei_3}    
    \includegraphics<5>[width=0.8\textwidth]{PLEO0483}       
  \end{center}
  \begin{itemize}[<+->]
  \item forest : $3.2 km^2$
  \item bush : $0.3 km^2$
  \item beach : $0.3 km^2$ 
  \item reef : $0.7 km^2$
  \end{itemize}
\end{frame}


\begin{frame}{The wandering tattler (\textit{Tringa incana}, Gmelin, 1789) }
  \begin{columns}
    \begin{column}[c]{0.6\textwidth}
      \begin{itemize}[<+->]
      \item Little shorebird about 110 g
      \item Breeding in the montain valley in North America {\tiny \cite{Gill2015}}
      \item Broadly distributed in the pacific area for wintering {\tiny \cite{GillJr.2002}}
      \item Seems to be territorial even during the wintering {\tiny
          \cite{Beichle2001}}\\
        {\footnotesize On Samoan islands Beichle records two birds share $700 m$ of coral
          beach ($350 m$ for each)}
      \end{itemize}
    \end{column}
    \begin{column}[c]{0.4\textwidth}
      \includegraphics<1>[width=\textwidth]{KiviKuaka_3_RL_20220306_155702_RL3_4503}
      \includegraphics<2>[width=\textwidth]{tattler_breeding}
      \includegraphics<3>[width=\textwidth]{tattler_wintering}
      \includegraphics<4>[width=0.8\textwidth]{KiviKuaka_3_RL_20220303_153324_RL3_4470}
    \end{column}
  \end{columns}
\end{frame}

    
\begin{frame} {Deployment of 17 PTT beacons}
   \begin{columns}
    \begin{column}[c]{0.6\textwidth}
      \begin{itemize}[<+->]
      \item We stayed on Tikei for 24h in January 2021
      \item During the day we estimated that 25 wandering tattlers lived on Tikei \\
        \footnotesize{(that corresponds to one bird for 350m coast too)}
      \item During the night,
        \begin{itemize}[<+->]
        \item we traped 17 birds mainly with bell-shaped net and powerfull ligth and some
          one with mistnet,
        \item we equipped them with a 5g Icarus PPT tag with a leg-loop harness.
        \end{itemize}
   \end{itemize}
     \end{column}
    \begin{column}[c]{0.4\textwidth}
      \begin{center}
        % \includegraphics<1-2>[width=.8\textwidth]{}
        % \includegraphics<3>[width=.8\textwidth]{bell-shaped_mistnet_mistnet}
        \includegraphics<5>[width=.8\textwidth]{KiviKuaka_3_RL_20210128_082356_RL3_2007}
         \end{center}
    \end{column}
  \end{columns}
\end{frame}



\begin{frame}{Icarus PTT beacon}
  \begin{columns}
    \begin{column}[c]{0.5\textwidth}
      \begin{itemize}[<+->]
      \item Max Plank initiative{\tiny \cite{Jetz2022}} 
      \item Little 5g GPS beacon with two antennas
      \item The beacons sent data by a VHF signal via a Russian antenna on the ISS
        \begin{itemize}
        \item ISS flights at low altitudes
        \item but they were some constraints on communication (such as
          the time windows)
        \end{itemize}
      \item The initial purpose of this project is to create a network for monitoring the movement of animals.   {\tiny \cite{Jetz2022}}
      \end{itemize}
         \end{column}
    \begin{column}[c]{0.5\textwidth}
      \begin{center}
        \includegraphics[width=0.7\textwidth]{icarus}
        \vspace{12pt}
        \includegraphics<1-2>[width=.8\textwidth]{KiviKuaka_3_RL_20220307_062331_RL3_4511}
        \includegraphics<3-5>[width=\textwidth]{Icarus_AutoE}
        \includegraphics<6->[width=\textwidth]{jetz_et_al_2022_icarus}
        \end{center}
     \end{column}
   \end{columns}
 \end{frame}

\begin{frame}{Icarus PTT beacon}
  \begin{center}
   \begin{alertblock}{}
     Unfortunately, since march 2022, the system no longer functions due to the cessation of collaboration between Max Planck and the Russian space agency in response to the military invasion. 
   \end{alertblock}
  \end{center}
 \end{frame}



\begin{frame}{Data gathering}
  \begin{columns}
    \begin{column}[c]{0.4\textwidth}
      \begin{itemize}[<+->]
   \item  17 wandering tattler fitted with GPS tags 
   \item  Only 13 have sent data
   \item  The data  acquisition was irregular and, unfortunately, stopped pretty quickly 
   \end{itemize}
    \end{column}
    \begin{column}[c]{0.6\textwidth}
      \begin{center}
        \includegraphics<1-2>[width=\textwidth]{Chevalier-errant-Tikei-Polynesie-francaise-scaled-e1618956657967-1536x1021}
        \includegraphics<3>[width=\textwidth]{Chevalier_errant_loc_per_day}
      \end{center}
    \end{column}
  \end{columns}
\end{frame}


\begin{frame}{Data gathering}
  \begin{center}
    \includegraphics<1>[width=\textwidth]{tikei_tattler}
    \includegraphics<2>[width=0.9\textwidth]{tikei_tattler_week}    
  \end{center}
\end{frame}


\begin{frame}{Data gathering}
  The data are fragmented but often synchronous thanks to the
  constraints of ISS time windows.
 \begin{center}
    \includegraphics[width=.7\textwidth]{synchronous_data}
  \end{center}
 \begin{block}<2>{}
       {\footnotesize 3310 pairs of synchronous data, with less than 2 minutes between the two locations}
 \end{block}
\end{frame}


% ===============================================
\section{Methods \& results}
% ===============================================



\begin{frame} {Bird home ranges}
  \begin{itemize}
  \item Location data are suitable for estimating home ranges (or areas used by individuals) {\tiny \cite{Kie2010}}
  \item To assess them we used the adehabitat R package {\tiny
      \cite{Calenge2015,Calenge2006} }
  \item As recommended for the small sample, we assess an individual 
    \textit{ad-hoc parameter H} for which the $95\%$ isoline forms a single simple polygon {\tiny \cite{Schuler2014}} 
  \end{itemize}
\end{frame}

 

\begin{frame}{Bird home ranges: trial-and-error agorithm}
    \begin{center}
     \includegraphics<1>[width=\textwidth]{get_h_accumulation_all}
     \includegraphics<2>[width=\textwidth]{get_h_kernel_T09_red}
      \end{center}
\end{frame}


\begin{frame}{Home range and core area}
  \begin{itemize}[<+->]
  \item  Isoline $95\%$ represents the home range of birds L'isoline $95\%$ représente le domaine vital des oiseaux 
  \item  Isoline $50\%$ isoline corresponds to their core of their home range, i.e. where they spend more than half of their time
    \tiny{\cite{Benhamou2013,Jourdan2021}}
  \end{itemize}
 \end{frame}


\begin{frame}{Home range and core area}
    \begin{columns}
    \begin{column}[c]{0.4\textwidth}
      \begin{itemize}[<+->]
      \item Except T06 the wanderind tattlers do not use the forest.
        \begin{itemize}[<+->]
        \item the reef
        \item the beach
        \item the bush
        \end{itemize}
      \item We intersect the home ranges with the used habitats
      \end{itemize}
    \end{column}
    \begin{column}[c]{0.6\textwidth}
      \begin{center}
        \includegraphics[width=\textwidth]{gg_bird_habitat-1}
      \end{center}
    \end{column}
  \end{columns}
\end{frame}



\begin{frame}{Home range}
  \begin{center}
     \includegraphics<1>[width=0.7\textwidth]{kernel_95_all}
   \end{center}
\end{frame}


\begin{frame}{Distances between birds}
  \begin{itemize}[<+->]
  \item we assess the distance between bird when the lag between the
    to location is less than 2 minutes
  \item we compare this distance with a control group with a lag
    comprise between 2 hours and 2 day. 
  \item To observe the effect of interaction levels between birds on these distances, we classify the pairs according to the minimum kernel at which the birds overlap. 
  \end{itemize}
\end{frame}


\begin{frame}{Distances between birds}
  \begin{center}
     \includegraphics[width=0.7\textwidth]{glmm_dist_kernel-time_explain}
   \end{center}
\end{frame}

\begin{frame}{Distances between birds}
  \begin{center}
     \includegraphics[width=0.7\textwidth]{glmm_dist_kernelTXT-time_day}
   \end{center}
\end{frame}

\begin{frame}{Distances between birds}
  \begin{center}
    \includegraphics<1>[width=0.7\textwidth]{glmm_dist_kernel-time}
    \includegraphics<2>[width=0.7\textwidth]{glmm_dist_kernel-time_cost}
   \end{center}
\end{frame}


% ===============================================
\section{Conclusion}
% ===============================================

\begin{frame}{Conclusions}
  \begin{itemize}
  \item les oiseaux semblent territoriaux
  \item les intéraction semblent limitées malgré le recouvrement des
    domaines vitaux
  \end{itemize}
\end{frame}




\begin{frame}[plain]
  \begin{center}
    \includegraphics[width=0.7\textwidth]{KiviKuaka_3_RL_20210209_172324_WhatsApp} \\
   % \vspace{12pt}
    \begin{Large}
      Thank you to the Kivikuaka team \\and the French navy
      for the field work, \\and thank you for your attention.
    \end{Large}
  \end{center}
\end{frame}


\begin{frame}[allowframebreaks]
  \begin{tiny}
    \frametitle{Réferences}
    \bibliographystyle{apalike}
    \bibliography{bib_files/biblio}
  \end{tiny}
\end{frame}




\end{document}
