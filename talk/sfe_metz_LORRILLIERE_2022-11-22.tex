\documentclass[10pt,compress]{beamer}
\usepackage[utf8]{inputenc}
%\usepackage[T1]{fontenc}
\usepackage[english]{babel}
% \usepackage{natbib}
%\usepackage{hyperref}
% \usepackage{animate}
%\usepackage{graphicx}
%\usepackage{color}

\pdfminorversion=5 
\pdfcompresslevel=9
\pdfobjcompresslevel=3

%% Theme
% \useinnertheme[shadow=true]{rounded}
% \useoutertheme{shadow}
% \usecolortheme{orchid}
% \usecolortheme{whale}%\usetheme{progressbar}
%\usetheme[]{Berlin}

\useoutertheme[footline=authortitle,subsection=false]{miniframes}
%\useoutertheme{smoothtree}
\useinnertheme[shadow=true]{rounded}
\usecolortheme{orchid}
\usecolortheme{whale}

\usepackage{etoolbox}
\makeatletter
\patchcmd{\slideentry}{\advance\beamer@tempdim by -.05cm}{\advance\beamer@tempdim by\beamer@vboxoffset\advance\beamer@tempdim by\beamer@boxsize\advance\beamer@tempdim by 1.2\pgflinewidth}{}{}
\patchcmd{\slideentry}{\kern\beamer@tempdim}{\advance\beamer@tempdim by 2pt\advance\beamer@tempdim by\wd\beamer@sectionbox\kern\beamer@tempdim}{}{}
\makeatother

\graphicspath{{./image/}} 

% \usecolortheme{progressbar}
 %\usefonttheme{progressbar}
 %\useinnertheme{progressbar}
 %\useoutertheme{progressbar}
%% Logo
% \logo{\includegraphics[height=0.5cm]{./images/Logo.png}}
%% Fond de la page
 %\setbeamercolor{background canvas}{bg=couleur}

\title[Wandering tattlers of Tikei]{Overlapping territories in a small wintering population of wandering tattler \\in French Polynesia}  

\author[Lorrillière \& Jiguet]{Romain Lorrillière
  \footnotesize{\textit{romain.lorrilliere@mnhn.fr}} \&  Frédéric Jiguet}
\institute{MNHN - CESCO}
\date{SFE$^2$-GfÖ-EEF Metz 2022-11-23}

\titlegraphic{
\includegraphics[width=\textwidth]{title-tattler}
}

%\AtBeginSection[]{
%  \begin{frame}
%   \frametitle{\insertsectionhead}
%    \scriptsize \tableofcontents[currentsection,hideothersubsections]
%  \end{frame}
%}



%\AtBeginSubsection[]{
%  \begin{frame}
%    \frametitle{\insertsubsectionhead}
%    \scriptsize \tableofcontents[sectionstyle=show/shaded,subsectionstyle=show/shaded/hide ]
%  \end{frame}
%}
%\addtobeamertemplate{footline}{\insertframenumber/\inserttotalframenumber}

\begin{document}
\maketitle



% \section{Sommaire}
%% =====================
%\begin{frame}
%   \frametitle{Sommaire}
%   \tableofcontents[hideallsubsections]
% \end{frame}




% ===============================================
\section{Introduction}
% ===============================================


\begin{frame}
  \frametitle{The shorebirds}
  \begin{columns}
    \begin{column}[c]{0.5\textwidth}
      \begin{center}
        \includegraphics<1->[width=\textwidth]{shorebird_murmuring_2}
      \end{center}
    \end{column}
    \begin{column}[c]{0.5\textwidth}
      \begin{center}
        \includegraphics<2>[width=\textwidth]{shorebird_foraging_2}     
      \end{center}
    \end{column}
  \end{columns}
\end{frame}

\begin{frame}{The shorebirds: a threatened taxon}
 \begin{columns}
    \begin{column}[c]{0.55\textwidth}
      \begin{itemize}[<+->]
      \item about $40 \%$ decline in species {\tiny \cite{Zoeckler2003,Studds2017}}
      \item many threats {\tiny \cite{Sutherland2012}}
        \begin{itemize}
        \item human threats \\{\footnotesize ex: climatic change,
            degradation and destruction of stopover shore habitat}
        \item next threats \\{\footnotesize ex:agricultural intensification, pollution and disease, eutrophication}
        \item natural threats \\{\footnotesize ex: tsunami,
            storm, volcanism, earthquake}
        \end{itemize}
      \end{itemize}
    \end{column}
    \begin{column}[c]{0.45\textwidth}
      \begin{center}
       \includegraphics[width=\textwidth]{limi}     
      \end{center}
    \end{column}
  \end{columns}
\end{frame}


\begin{frame}{Stopover and wintering are crucial times for conservation}
  \begin{columns}
    \begin{column}[c]{0.60\textwidth}
      \begin{itemize}[<+->]
      \item The shorebird are long-distance migratory species
      \item The nutritional quality of stopover sites {\tiny
          \cite{Morrison2007,Studds2017}} and wintering areas {\tiny
          \cite{Piersma1993,Tulp2009}}  are major drivers to productivity
      \item In tropical area the avalaibles ressources could be limited
      \item These limitations drive competition that is the major
        driver to the shorebird survival during the winter {\tiny \cite{Baker1973}}
      \end{itemize}
    \end{column}
    \begin{column}[c]{0.4\textwidth}
      \begin{center}
        \includegraphics<1>[width=\textwidth]{migration_godwit}
        \includegraphics<2>[width=\textwidth]{feeding}
        \includegraphics<3->[width=\textwidth]{tropic}
      \end{center}
    \end{column}
  \end{columns}
\end{frame}


\begin{frame}{Interactions}
  \begin{columns}
    \begin{column}[c]{0.6\textwidth}
      \begin{itemize}[<+->]
      \item Territorial behaviours on the wintering areas
      \item Especially for species that visual feeders detected prey over large areas\\ {\footnotesize
          plovers, sandpipers and tattlers} {\tiny \cite{Colwell2000}}
      \end{itemize}
    \end{column}
    \begin{column}[c]{0.4\textwidth}
      \includegraphics<1>[width=\textwidth]{fight}
      \includegraphics<2>[width=0.7\textwidth]{plover-sandpiper-tattler}     
    \end{column}
  \end{columns}
\end{frame}


\begin{frame}{Many studies about wintering on mudflat and wetland }
  \begin{columns}
    \begin{column}[c]{0.55\textwidth}
      \begin{itemize}[<+->]
      \item Many studies about wintering on mudflat and wetland \\
        {\footnotesize Europe, Asia, South America, Australia,
          New-Zeland}
        \begin{itemize}
        \item tidal effects on habitat using
        \item polution effects on feed productivity
        \item interaction effects between many species in one place
        \end{itemize}
      \end{itemize}
    \end{column}
    \begin{column}[c]{0.45\textwidth}
      \includegraphics[width=\textwidth]{KiviKuaka_3_RL_20140912_071004_P9120201}
      \end{column}
  \end{columns}
\end{frame}

\begin{frame}{Many studies about wintering on mudflat and wetland }
  \begin{columns}
    \begin{column}[c]{0.55\textwidth}
      \begin{itemize}[<+->]
      \item In the Pacific Ocean, despite the absence of these productive habitats on the many reef islands, there are wintering birds
      \item For conservation issue it seems to be important to
        understand how the birds interact on these reef island for food
       \end{itemize}
    \end{column}
    \begin{column}[c]{0.45\textwidth}
      \includegraphics[width=\textwidth]{KiviKuaka_3_RL_20210127_124920_P1270001}
     \end{column}
  \end{columns}
\end{frame}



\begin{frame}{Our main questions}
  What is the spatial structure of a territorial shorebird population on a reef island?  
  \begin{itemize}
  \item Probably important overlap of bird home ranges 
  \item But despite the overlap, the birds should stay away from each other..
 \end{itemize}
\end{frame}




% ===============================================
\section{Material et methods}
% ===============================================


 
\begin{frame}{Tikei: a tiny isolated island}
  \begin{center}
     \includegraphics<1-2>[width=0.6\textwidth]{polynesie_sites_1}   
     \includegraphics<3-4>[width=0.6\textwidth]{Tikei_2}
     \includegraphics<5-6>[width=0.6\textwidth]{KiviKuaka_3_RL_20210128_032720_DJI_0559-1} 
  \end{center}
  \begin{itemize}[<+->]
  \item Little island of Tuamotu archipelago in French Polynesia    
  \item $70 km$ to the closest islands
  \item $3.9 km$ in lenght and $1.6 km$ in width
  \item $4.5 km^2$  
  \item altitude max $3 m$
  \item $9 km$ de côte
  \end{itemize}
\end{frame}

\begin{frame}{Tikei: a tiny isolated island}
  \begin{center}
    \includegraphics<1-4>[width=0.6\textwidth]{Tikei_3}    
    \includegraphics<5>[width=0.8\textwidth]{PLEO0483}       
  \end{center}
  \begin{itemize}[<+->]
  \item forest : $3.2 km^2$
  \item bush : $0.3 km^2$
  \item beach : $0.3 km^2$ 
  \item reef : $0.7 km^2$
  \end{itemize}
\end{frame}


\begin{frame}{The wandering tattler (\textit{Tringa incana}, Gmelin, 1789) }
  \begin{columns}
    \begin{column}[c]{0.6\textwidth}
      \begin{itemize}[<+->]
      \item Little shorebird about 110 g
      \item Breeding in the montain valley in North America {\tiny \cite{Gill2015}}
      \item Broadly distributed in the pacific area for wintering {\tiny \cite{GillJr.2002}}
      \item Seems to be territorial even during the wintering {\tiny
          \cite{Beichle2001}}\\
        {\footnotesize On Samoan islands Beichle records two birds share $700 m$ of coral
          beach ($350 m$ for each)}
      \end{itemize}
    \end{column}
    \begin{column}[c]{0.4\textwidth}
      \includegraphics<1>[width=\textwidth]{KiviKuaka_3_RL_20220306_155702_RL3_4503}
      \includegraphics<2>[width=\textwidth]{tattler_breeding}
      \includegraphics<3>[width=\textwidth]{tattler_wintering}
      \includegraphics<4>[width=0.8\textwidth]{KiviKuaka_3_RL_20220303_153324_RL3_4470}
    \end{column}
  \end{columns}
\end{frame}

    
\begin{frame} {Déployment des balises}
   \begin{columns}
    \begin{column}[c]{0.6\textwidth}
   \begin{itemize}
   \item Among about 25 wandering tattlers living on Tikei \\
     \footnotesize{(about one bird for 350m cost too)}
   \item During one night in January 2021, we equiped 17 birds with a
     Icarus PPT tag. 
   \end{itemize}
     \end{column}
    \begin{column}[c]{0.4\textwidth}
   \begin{center}
     \includegraphics<1>[width=.8\textwidth]{KiviKuaka_3_RL_20220307_062331_RL3_4511}
     \includegraphics<2>[width=.8\textwidth]{KiviKuaka_3_RL_20210128_082356_RL3_2007}
   \end{center}
    \end{column}
  \end{columns}
\end{frame}



\begin{frame}{Les balises : la contrainte du poids}
  \begin{columns}
    \begin{column}[c]{0.55\textwidth}
      \begin{itemize}
      \item effet des balises sur la survie et la productivité dès
        $1\%$ du poids de l'oiseau {\tiny \cite{Bodey2018}}
      \item règle éthique : pas plus de $5\%$ \\au USA $3\%$ {\tiny \cite{Bridge2011}}
      \end{itemize}
    \end{column}
    \begin{column}[c]{0.45\textwidth}
          \includegraphics[width=\textwidth]{Bridge_et_al_2011_logger}       
    \end{column}
  \end{columns}
\end{frame}


\begin{frame}{Les balises Icarus}
  \begin{columns}
    \begin{column}[c]{0.5\textwidth}
      \begin{itemize}[<+->]
      \item initiative du Max Plank {\tiny \cite{Jetz2022}} 
      \item petite balise GPS de 5g
      \item communique via ISS par VHF
        \begin{itemize}
        \item basse altitude
        \item contrainte sur les aquisitions
        \end{itemize}
      \item objectif de réseau de surveillance {\tiny \cite{Jetz2022}}
      \end{itemize}
         \end{column}
    \begin{column}[c]{0.5\textwidth}
      \begin{center}
        \includegraphics[width=0.5\textwidth]{icarus}
        \vspace{12pt}
        \includegraphics<3-5>[width=\textwidth]{Icarus_AutoE}
        \includegraphics<6->[width=\textwidth]{jetz_et_al_2022_icarus}
       \end{center}
     \end{column}
  \end{columns}
 \end{frame}




 \begin{frame} {Déployment des balises}
   \begin{itemize}
   \item  17 chevaliers ont été équipés du 27 au 28 janvier 2021
   \item  13 balises ont envoyées des données
   \end{itemize}
   \begin{center}
     \includegraphics<1>[width=0.8\textwidth]{Chevalier-errant-Tikei-Polynesie-francaise-scaled-e1618956657967-1536x1021}
     \includegraphics<2>[width=0.8\textwidth]{tikei_tattler}
   \end{center}
\end{frame}

\begin{frame}{L'utilisation des habitats}
   \begin{columns}
    \begin{column}[c]{0.4\textwidth}
      à l'exception de T06 les chevaliers n'utilisent pas la forêt et utilise que :
      \begin{itemize}
      \item le platier
      \item la plage
      \item les broussailles
      \end{itemize}
    \end{column}
    \begin{column}[c]{0.6\textwidth}
      \begin{center}
        \includegraphics[width=\textwidth]{gg_bird_habitat-1}
      \end{center}
    \end{column}
  \end{columns}
\end{frame}


\begin{frame}{L'aquisition des données}
  \begin{columns}
    \begin{column}[c]{0.4\textwidth}
      Une aquisition irrégulière et qui s'arrète assez vite :-/
    \end{column}
    \begin{column}[c]{0.6\textwidth}
      \begin{center}
        \includegraphics[width=\textwidth]{Chevalier_errant_loc_per_day}
      \end{center}
    \end{column}
  \end{columns}
\end{frame}


\begin{frame}{L'aquisition des données}
  Une aquisition irrégulière et qui s'arrète assez vite :-/
  \begin{center}
    \includegraphics[width=0.8\textwidth]{tikei_tattler_week}    
  \end{center}
\end{frame}


\begin{frame}{L'aquisition des données}
  \begin{columns}
    \begin{column}[c]{0.4\textwidth}
      Une aquisition qui semble dépendre des passages de l'ISS
    \end{column}
    \begin{column}[c]{0.6\textwidth}
      \begin{center}
        \includegraphics[width=\textwidth]{Tringa_incana_loc_per_day}
      \end{center}
    \end{column}
  \end{columns}
\end{frame}


\begin{frame}{L'aquisition des données}
  Mais des données relativement synchrones :-)\\
  \begin{center}
    \includegraphics[width=\textwidth]{tikei_tattler_nb_data_week_day_hour}
  \end{center}
\end{frame}

\begin{frame}{L'aquisition des données}
  Mais des données relativement synchrones :-)\\
  \begin{center}
    \includegraphics[width=\textwidth]{tikei_tattler_nb_data_week_day_hour}
  \end{center}
    {\footnotesize 4647 couple de données séparé de moins de 15 minutes}
\end{frame}


\begin{frame} {Domainee vitaux}
  \begin{itemize}
  \item Données de localisation sont adaptées à l'estimation des
    domaines vitaux (ou zones utilisées par les indivudus) {\tiny \cite{Kie2010}}
  \item on les calcule avec le package adehabitat {\tiny \cite{Calenge2015,Calenge2006} }
  \end{itemize}
\end{frame}


\begin{frame}{Kernel paramétrage : h ?}
  \begin{itemize}
  \item La fonction de kernel \textit{kernelUD} a un paramètre d'ajustement
    $h$. Par defaut sa valeur \textit{h-ref} est adapté au jeux de
    données qui ont beaucoup de localisations distribuées de manière unimodale. Mais dans le cas
    d'échantillonnage petit ce paramètre va surestimé la taille du domaine
    vital {\tiny \cite{Schuler2014}}.
  \item dans les cas de faible échantillonnage il est suggéré de
    réaliser une recherche d'un \textit{h ad hoc } {\tiny \cite{Schuler2014}}
  \end{itemize}
\end{frame}


\begin{frame}{Kernel paramétrage : h ?}
  Algorithme simple
  \begin{itemize}
  \item pour chaque oiseau
  \item recherche de l'\textit{extent} et du plus petit $h$ pour lequel l'isoligne
    $95\%$ forme un seul polygon simple
  \item algorithme de recherche est un processus simple  essai / erreur
  \end{itemize}
\end{frame}


\begin{frame}{Kernel paramétrage : h ?}
  \begin{center}
     \includegraphics<1>[width=\textwidth]{get_h_accumulation_all}
     \includegraphics<2>[width=\textwidth]{get_h_kernel_T09_red}
     \includegraphics<3>[width=\textwidth]{get_h_kernel_T19_red}
     \includegraphics<4>[width=\textwidth]{get_h_kernel_T22_red}
  \end{center}
\end{frame}


\begin{frame}{Domaine vital et coeur d'air}
  \begin{itemize}[<+->]
  \item   L'isoline $95\%$ représente le domaine vital des oiseaux 
  \item L'isoline $50\%$ correspond au couer de leur domaine vital
    c'est à dire où ils passent plus de la moitier de leur temps
    \tiny{\cite{Benhamou2013,Jourdan2021}} 
  \item intersections avec les habitats utilisés
    \begin{itemize}
    \item le platier
    \item la plage
    \item les brousaille
    \end{itemize}
  \end{itemize}
\end{frame}


\begin{frame}{Domaines vitaux et coeur d'air}
  \begin{center}
     \includegraphics<1>[width=\textwidth]{tikei_tattler_kernel_T09_red}
     \includegraphics<2>[width=\textwidth]{tikei_tattler_kernel_T19_red}
     \includegraphics<3>[width=\textwidth]{tikei_tattler_kernel_T22_red}
  \end{center}
\end{frame}



% ===============================================
\section{Résultats}
% ===============================================

\begin{frame}{Domaine vital et coeur d'air}
  \begin{center}
     \includegraphics<1>[width=0.7\textwidth]{kernel_95_all}
     \includegraphics<2>[width=\textwidth]{intersect_95_50}
   \end{center}
\end{frame}


\begin{frame}{Distances entre individus}
  \begin{center}
     \includegraphics<1>[width=0.8\textwidth]{dis_violin_jour_nuit_kernel}
     \includegraphics<2>[width=0.8\textwidth]{dis_time_jour_nuit_kernel}
      \includegraphics<3>[width=0.9\textwidth]{glm}
   \end{center}
\end{frame}


% ===============================================
\section{Conclusion}
% ===============================================

\begin{frame}{Conclusions}
  \begin{itemize}
  \item les oiseaux semblent territoriaux
  \item les intéraction semblent limitées malgré le recouvrement des
    domaines vitaux
  \end{itemize}
\end{frame}




\begin{frame}[plain]
  \begin{center}
    \includegraphics[width=0.8\textwidth]{KiviKuaka_3_RL_20210209_172324_WhatsApp} \\
   % \vspace{12pt}
    \begin{Large}
      Merci à l'équipe Kivikuaka   
    \end{Large}
  \end{center}
\end{frame}


\begin{frame}[allowframebreaks]
  \begin{tiny}
    \frametitle{Réferences}
    \bibliographystyle{apalike}
    \bibliography{bib_files/biblio}
  \end{tiny}
\end{frame}




\end{document}
