\documentclass[10pt,compress]{beamer}
\usepackage[utf8]{inputenc}
%\usepackage[T1]{fontenc}
\usepackage[english]{babel}
% \usepackage{natbib}
%\usepackage{hyperref}
% \usepackage{animate}
%\usepackage{graphicx}
%\usepackage{color}

\pdfminorversion=5 
\pdfcompresslevel=9
\pdfobjcompresslevel=3

%% Theme
% \useinnertheme[shadow=true]{rounded}
% \useoutertheme{shadow}
% \usecolortheme{orchid}
% \usecolortheme{whale}%\usetheme{progressbar}
%\usetheme[]{Berlin}

\useoutertheme[footline=authortitle,subsection=false]{miniframes}
%\useoutertheme{smoothtree}
\useinnertheme[shadow=true]{rounded}
\usecolortheme{orchid}
\usecolortheme{whale}

\usepackage{etoolbox}
\makeatletter
\patchcmd{\slideentry}{\advance\beamer@tempdim by -.05cm}{\advance\beamer@tempdim by\beamer@vboxoffset\advance\beamer@tempdim by\beamer@boxsize\advance\beamer@tempdim by 1.2\pgflinewidth}{}{}
\patchcmd{\slideentry}{\kern\beamer@tempdim}{\advance\beamer@tempdim by 2pt\advance\beamer@tempdim by\wd\beamer@sectionbox\kern\beamer@tempdim}{}{}
\makeatother

\graphicspath{{./image/}} 

% \usecolortheme{progressbar}
 %\usefonttheme{progressbar}
 %\useinnertheme{progressbar}
 %\useoutertheme{progressbar}
%% Logo
% \logo{\includegraphics[height=0.5cm]{./images/Logo.png}}
%% Fond de la page
 %\setbeamercolor{background canvas}{bg=couleur}

\title[Wandering tattlers of Tikei]{Overlapping territories in a small wintering population of wandering tattler \\in French Polynesia}  

\author[Lorrillière \& Jiguet]{Romain Lorrillière
  \footnotesize{\textit{romain.lorrilliere@mnhn.fr}} \&  Frédéric Jiguet}
\institute{MNHN - CESCO}
\date{SFE$^2$-GfÖ-EEF Metz 2022-11-23}

\titlegraphic{
\includegraphics[width=\textwidth]{title-tattler}
}

%\AtBeginSection[]{
%  \begin{frame}
%   \frametitle{\insertsectionhead}
%    \scriptsize \tableofcontents[currentsection,hideothersubsections]
%  \end{frame}
%}



%\AtBeginSubsection[]{
%  \begin{frame}
%    \frametitle{\insertsubsectionhead}
%    \scriptsize \tableofcontents[sectionstyle=show/shaded,subsectionstyle=show/shaded/hide ]
%  \end{frame}
%}
%\addtobeamertemplate{footline}{\insertframenumber/\inserttotalframenumber}

\begin{document}
\maketitle



% \section{Sommaire}
%% =====================
%\begin{frame}
%   \frametitle{Sommaire}
%   \tableofcontents[hideallsubsections]
% \end{frame}




% ===============================================
\section{Introduction}
% ===============================================


\begin{frame}
  \frametitle{The shorebirds}
  \begin{columns}
    \begin{column}[c]{0.5\textwidth}
      \begin{center}
        \includegraphics<1->[width=\textwidth]{shorebird_murmuring_2}
      \end{center}
    \end{column}
    \begin{column}[c]{0.5\textwidth}
      \begin{center}
        \includegraphics<2>[width=\textwidth]{shorebird_foraging_2}     
      \end{center}
    \end{column}
  \end{columns}
\end{frame}

\begin{frame}{The shorebirds: a threatened taxon}
 \begin{columns}
    \begin{column}[c]{0.55\textwidth}
      \begin{itemize}[<+->]
      \item About $40 \%$ decline in species {\tiny \cite{Zoeckler2003,Studds2017}}
      \item Many threats {\tiny \cite{Sutherland2012}}
        \begin{itemize}
        \item The human-induced threats \\{\footnotesize ex: climatic change,
            degradation and destruction of stopover shore habitat, agricultural intensification, pollution and disease, eutrophication}
        \item The natural threats \\{\footnotesize ex: tsunamis,
            storm, volcanism, earthquake}
        \end{itemize}
      \end{itemize}
    \end{column}
    \begin{column}[c]{0.45\textwidth}
      \begin{center}
       \includegraphics[width=\textwidth]{limi}     
      \end{center}
    \end{column}
  \end{columns}
\end{frame}


\begin{frame}{Stopover and wintering are crucial times for conservation}
  \begin{columns}
    \begin{column}[c]{0.60\textwidth}
      \begin{itemize}[<+->]
      \item The shorebirds are long-distance migratory species
      \item The nutritional quality of stopover sites {\tiny
          \cite{Morrison2007,Studds2017}} and wintering areas {\tiny
          \cite{Piersma1993,Tulp2009}}  are major drivers to productivity
      \item In tropical area the avalaibles ressources could be limited
      \item These limitations drive competition that is the major
        driver to the shorebird survival during the winter {\tiny \cite{Baker1973}}
      \end{itemize}
    \end{column}
    \begin{column}[c]{0.4\textwidth}
      \begin{center}
        \includegraphics<1>[width=\textwidth]{migration_godwit}
        \includegraphics<2>[width=\textwidth]{feeding}
        \includegraphics<3->[width=\textwidth]{tropic}
      \end{center}
    \end{column}
  \end{columns}
\end{frame}


\begin{frame}{Interactions}
  \begin{columns}
    \begin{column}[c]{0.6\textwidth}
      \begin{itemize}[<+->]
      \item Territorial behaviours on the wintering areas
      \item Especially for species that visual feeders detected prey over large areas\\ {\footnotesize
          plovers, sandpipers and tattlers} {\tiny \cite{Colwell2000}}
      \end{itemize}
    \end{column}
    \begin{column}[c]{0.4\textwidth}
      \includegraphics<1>[width=\textwidth]{fight}
      \includegraphics<2>[width=0.7\textwidth]{plover-sandpiper-tattler}     
    \end{column}
  \end{columns}
\end{frame}


\begin{frame}{Many studies about wintering on mudflat and wetland }
  \begin{columns}
    \begin{column}[c]{0.55\textwidth}
      \begin{itemize}[<+->]
      \item Many studies about wintering on mudflats and wetlands \\
        {\footnotesize Europe, Asia, South America, Australia,
          New Zealand}
        \begin{itemize}
          \item tidal effects on habitat used
        \item pollution effects on feed resources productivity
        \item interaction effects between the many species that feed on the mudflats
        \end{itemize}
      \end{itemize}
    \end{column}
    \begin{column}[c]{0.45\textwidth}
      \includegraphics[width=\textwidth]{KiviKuaka_3_RL_20140912_071004_P9120201}
      \end{column}
  \end{columns}
\end{frame}

\begin{frame}{Many studies about wintering on mudflat and wetland }
  \begin{columns}
    \begin{column}[c]{0.55\textwidth}
      \begin{itemize}[<+->]
      \item Despite the absence of these productive habitats on the many reef islands in the pacific ocean, there are wintering shorebirds.
      \item For conservation issues, it seems essential to understand how the birds interact on these reefed islands for feed.
       \end{itemize}
    \end{column}
    \begin{column}[c]{0.45\textwidth}
      \includegraphics[width=\textwidth]{KiviKuaka_3_RL_20210127_124920_P1270001}
     \end{column}
  \end{columns}
\end{frame}



\begin{frame}{Our main questions}
  What is the spatial structure of a territorial shorebird population on a reef island?  
  \begin{itemize}
  \item Probably important overlap of bird home ranges 
  \item But despite the overlap, the birds should stay away from each other..
 \end{itemize}
\end{frame}




% ===============================================
\section{Material}
% ===============================================


 
\begin{frame}{Tikei: a tiny isolated island}
  \begin{center}
     \includegraphics<1-2>[width=0.6\textwidth]{polynesie_sites_1}   
     \includegraphics<3-4>[width=0.6\textwidth]{Tikei_2}
     \includegraphics<5-6>[width=0.6\textwidth]{KiviKuaka_3_RL_20210128_032720_DJI_0559-1} 
  \end{center}
  \begin{itemize}[<+->]
  \item Little island of Tuamotu archipelago in French Polynesia    
  \item $70 km$ to the closest islands
  \item $3.9 km$ in length and $1.6 km$ in width
  \item $4.5 km^2$  
  \item altitude max $3 m$
  \item $9 km$ de côte
  \end{itemize}
\end{frame}

\begin{frame}{Tikei: a tiny isolated island}
  \begin{center}
    \includegraphics<1-4>[width=0.6\textwidth]{Tikei_3}    
    \includegraphics<5>[width=0.8\textwidth]{PLEO0483}       
  \end{center}
  \begin{itemize}[<+->]
  \item forest : $3.2 km^2$
  \item bush : $0.3 km^2$
  \item beach : $0.3 km^2$ 
  \item reef : $0.7 km^2$
  \end{itemize}
\end{frame}


\begin{frame}{The wandering tattler (\textit{Tringa incana}, Gmelin, 1789) }
  \begin{columns}
    \begin{column}[c]{0.6\textwidth}
      \begin{itemize}[<+->]
      \item Small shorebird about 110 g
      \item Breeding in the mountain valley in North America {\tiny \cite{Gill2015}}
      \item Broadly distributed in the pacific area for wintering {\tiny \cite{GillJr.2002}}
      \item Seems to be territorial even during the wintering {\tiny
          \cite{Beichle2001}}\\
        {\footnotesize On Samoan islands Beichle records two birds share $700 m$ of coral
          beach ($350 m$ for each)}
      \end{itemize}
    \end{column}
    \begin{column}[c]{0.4\textwidth}
      \includegraphics<1>[width=\textwidth]{KiviKuaka_3_RL_20220306_155702_RL3_4503}
      \includegraphics<2>[width=\textwidth]{tattler_breeding}
      \includegraphics<3>[width=\textwidth]{tattler_wintering}
      \includegraphics<4>[width=0.8\textwidth]{KiviKuaka_3_RL_20220303_153324_RL3_4470}
    \end{column}
  \end{columns}
\end{frame}

    
\begin{frame} {Deployment of 17 PTT beacons}
   \begin{columns}
    \begin{column}[c]{0.6\textwidth}
      \begin{itemize}[<+->]
      \item We stayed on Tikei for 24h in January 2021
      \item During the day, we estimated that 25 wandering tattlers lived on Tikei
 \\
        \footnotesize{(that corresponds to one bird for 350m coast too)}
      \item During the night, We traped 17 birds, mainly with bell-shaped nets and powerful light and someone with mist nets
        \item We equipped them with a 5g Icarus PPT tag with a leg-loop harness.
      
   \end{itemize}
     \end{column}
    \begin{column}[c]{0.4\textwidth}
      \begin{center}
        \includegraphics<1-2>[width=.8\textwidth]{RL3_1280}
        \includegraphics<3>[width=.8\textwidth]{capture}
        \includegraphics<4>[width=.8\textwidth]{KiviKuaka_3_RL_20210128_082356_RL3_2007}
         \end{center}
    \end{column}
  \end{columns}
\end{frame}



\begin{frame}{Icarus PTT beacon}
  \begin{columns}
    \begin{column}[c]{0.5\textwidth}
      \begin{itemize}[<+->]
      \item Max Plank initiative{\tiny \cite{Jetz2022}} 
      \item The beacons sent data by a VHF signal via a Russian antenna on the ISS
        \begin{itemize}
        \item ISS flights at low altitudes
        \item but they were some constraints on communication (such as
          the time windows)
        \end{itemize}
      \end{itemize}
         \end{column}
    \begin{column}[c]{0.5\textwidth}
      \begin{center}
        \includegraphics[width=0.7\textwidth]{icarus}
        \vspace{12pt}
        \includegraphics<1-2>[width=.8\textwidth]{KiviKuaka_3_RL_20220307_062331_RL3_4511}
        \includegraphics<3->[width=\textwidth]{Icarus_AutoE}
        \end{center}
     \end{column}
   \end{columns}
 \end{frame}

\begin{frame}{Icarus PTT beacon}
  \begin{center}
   \begin{alertblock}{}
     Unfortunately, since march 2022, the system no longer works due to the cessation of collaboration between Max Planck and the Russian space agency in response to the military invasion. 
   \end{alertblock}
  \end{center}
 \end{frame}



\begin{frame}{Data gathering}
  \begin{columns}
    \begin{column}[c]{0.4\textwidth}
      \begin{itemize}[<+->]
   \item  17 wandering tattler fitted with GPS tags 
   \item  Only 13 have sent data
   \item  The data  acquisition was irregular and, unfortunately, stopped pretty quickly 
   \end{itemize}
    \end{column}
    \begin{column}[c]{0.6\textwidth}
      \begin{center}
        \includegraphics<1-2>[width=\textwidth]{Chevalier-errant-Tikei-Polynesie-francaise-scaled-e1618956657967-1536x1021}
        \includegraphics<3>[width=\textwidth]{Chevalier_errant_loc_per_day}
      \end{center}
    \end{column}
  \end{columns}
\end{frame}


\begin{frame}{Data gathering}
  \begin{center}
    \includegraphics<1>[width=\textwidth]{tikei_tattler}
    \includegraphics<2>[width=0.9\textwidth]{tikei_tattler_week}    
  \end{center}
\end{frame}


\begin{frame}{Data gathering}
  The data are fragmented but often synchronous thanks  the
  constraints of ISS time windows.
 \begin{center}
    \includegraphics[width=.7\textwidth]{synchronous_data}
  \end{center}
 \begin{block}<2>{}
       {\footnotesize 3310 pairs of synchronous data, with less than 2 minutes between the two locations}
 \end{block}
\end{frame}


% ===============================================
\section{Methods \& results}
% ===============================================



\begin{frame} {Bird home ranges}
  \begin{itemize}
  \item Location data are suitable for estimating home ranges (or areas used by individuals) {\tiny \cite{Kie2010}}
  \item To assess them we used the adehabitat R package {\tiny
      \cite{Calenge2015,Calenge2006} }
  \item As recommended for the small sample, we assess an individual 
    \textit{ad-hoc parameter H} for which the $95\%$ isoline forms a single simple polygon {\tiny \cite{Schuler2014}} 
  \end{itemize}
\end{frame}

 

\begin{frame}{Bird home ranges: trial-and-error algorithm}
    \begin{center}
     \includegraphics<1>[width=\textwidth]{get_h_accumulation_all}
     \includegraphics<2>[width=\textwidth]{get_h_kernel_T09_red}
      \end{center}
\end{frame}


\begin{frame}{Home range and core area}
    \begin{columns}
    \begin{column}[c]{0.4\textwidth}
      \begin{itemize}[<+->]
      \item We improve home ranges that are smooth on location according to the habitats used
      \item Except T06 the wanderind tattlers do not use the forest.
      \item We intersect the home ranges with the used habitats
         \begin{itemize}
        \item the reef
        \item the beach
        \item the bush
        \end{itemize}
      \end{itemize}
    \end{column}
    \begin{column}[c]{0.6\textwidth}
      \begin{center}
        \includegraphics[width=\textwidth]{gg_bird_habitat-1}
      \end{center}
    \end{column}
  \end{columns}
\end{frame}



\begin{frame}{Home range}
  \begin{center}
     \includegraphics<1>[width=0.7\textwidth]{kernel_95_all}
   \end{center}
\end{frame}


\begin{frame}{Distances between birds}
  \begin{itemize}[<+->]
  \item We assess the distance between birds when the lag between the locations is less than 2 minutes.
  \item We compare this distance with a control group with a lag from 2 hours to 2 days.
  \item To observe the effect of interaction levels between birds on these distances, we classify the pairs according to the minimum isoline at which the birds overlap. 
  \end{itemize}
\end{frame}


\begin{frame}{Distances between birds}
  \begin{center}
     \includegraphics[width=0.7\textwidth]{glmm_dist_kernel-time_explain}
   \end{center}
\end{frame}

\begin{frame}{Distances between birds}
  \begin{center}
    \includegraphics<1>[width=0.7\textwidth]{glmm_dist_kernel-time}
   \end{center}
\end{frame}


\begin{frame}{Distances between birds}
  \begin{columns}
    \begin{column}[c]{0.5\textwidth}
      \begin{footnotesize}
        
      \begin{itemize}[<+->]
      \item Birds are, in mean, not so close. 
      \item Even when their home range overlap greatly birds are not
        close
      \item Birds seem to be not social
      \item But when the birds overlap significantly, they seem closer when locations are synchronous rather than randomised locations.
      \item This difference could be the result of the behaviour of territoriality, vigilance and aggressivity among the other bird.
      \item Interestingly, this difference seems to be an indicator of the cost of the competition. 
      \end{itemize}
    \end{footnotesize}
    \end{column}
    \begin{column}[c]{0.5\textwidth}
      \begin{center}
        \includegraphics<1-4>[width=\textwidth]{glmm_dist_kernel-time}
        \includegraphics<5->[width=\textwidth]{glmm_dist_kernel-time_cost}
      \end{center}
    \end{column}
  \end{columns}
\end{frame}


% ===============================================
\section{Conclusion}
% ===============================================

\begin{frame}{Conclusions 1/2}
  \begin{itemize}
  \item Despite a large overlap in home range, the Tikei wandering tattler appears to be territorial. 
  \item But this overlap of home ranges seems to come at a cost to the birds, probably through the time spent watching and defending their territory.
  \end{itemize}
\end{frame}

\begin{frame}{Conclusions 2/2}
  \begin{itemize}
  \item Despite fragmented data on the location of the birds
  \item we show an interesting example of birds interacting on a small isolated island.
  \end{itemize}
\end{frame}



\begin{frame}[plain]
  \begin{center}
    \includegraphics[width=0.7\textwidth]{thanks} \\
    \vspace{6pt}
   
      Thank to the Kivikuaka team \\and the French navy
      for the field work, \\and thank you for your attention.
   
  \end{center}
\end{frame}


\begin{frame}[allowframebreaks]
  \begin{tiny}
    \frametitle{Réferences}
    \bibliographystyle{apalike}
    \bibliography{bib_files/biblio}
  \end{tiny}
\end{frame}
 



\end{document}
