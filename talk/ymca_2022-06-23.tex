\documentclass[10pt]{beamer}
\usepackage[utf8]{inputenc}
%\usepackage[T1]{fontenc}
\usepackage[french]{babel}
% \usepackage{natbib}
%\usepackage{hyperref}
% \usepackage{animate}
%\usepackage{graphicx}
%\usepackage{color}

\pdfminorversion=5 
\pdfcompresslevel=9
\pdfobjcompresslevel=3

%% Theme
% \useinnertheme[shadow=true]{rounded}
% \useoutertheme{shadow}
% \usecolortheme{orchid}
% \usecolortheme{whale}%\usetheme{progressbar}
%\usetheme[]{Berlin}

\useoutertheme[footline=authorinstitutetitle]{miniframes}
\useoutertheme{smoothtree}
\useinnertheme[shadow=true]{rounded}
\usecolortheme{orchid}
\usecolortheme{whale}


\graphicspath{{./image/}} 

% \usecolortheme{progressbar}
% \usefonttheme{progressbar}
% \useinnertheme{progressbar}
% \useoutertheme{progressbar}
%% Logo
% \logo{\includegraphics[height=0.5cm]{./images/Logo.png}}
%% Fond de la page
% \setbeamercolor{background canvas}{bg=couleur}

\title[Chevaliers errants de Tikei]{Organisation spatiale d'une population de chevalier errant hivernante sur une petite ile polynésienne \\
  \textit{\footnotesize{Tribulation d'analyse d'un jeux de données}}}

\author{Romain Lorrilliere, Frédéric Jiguet | Projet Kivi Kuaka}
\date{YMCA 23 juin 2022}

\AtBeginSection[]{
  \begin{frame}
   \frametitle{\insertsectionhead}
    \scriptsize \tableofcontents[currentsection,hideothersubsections]
  \end{frame}
}



%\AtBeginSubsection[]{
%  \begin{frame}
%    \frametitle{\insertsubsectionhead}
%    \scriptsize \tableofcontents[sectionstyle=show/shaded,subsectionstyle=show/shaded/hide ]
%  \end{frame}
%}
%\addtobeamertemplate{footline}{\insertframenumber/\inserttotalframenumber}

\begin{document}
\maketitle



% \section{Sommaire}
%% =====================
\begin{frame}
   \frametitle{Sommaire}
   \tableofcontents[hideallsubsections]
 \end{frame}




% ===============================================
\section{Introduction}
% ===============================================

\begin{frame}{Les limicoles taxon en surcis}
 \begin{columns}
    \begin{column}[c]{0.55\textwidth}
      \begin{itemize}[<+->]
      \item env. $40 \%$ des espèces en déclins {\tiny \cite{Zoeckler2003,Studds2017}}
      \item nombreuses causes de déclins {\tiny \cite{Sutherland2012}}
        \begin{itemize}
        \item risques naturels \\{\footnotesize ex: tsunami,
          tempète, volcanisme, tremblement de terre}
        \item risques anthropogéniques \\{\footnotesize ex: changement climatique,
          modification des vasières de halte migratoire, dégradation
          des habitat}
        \item risques futurs \\{\footnotesize ex:intensification agricole, polutions et
          maladies, eutrophisation}
        \end{itemize}
      \end{itemize}
    \end{column}
    \begin{column}[c]{0.45\textwidth}
      \begin{center}
       \includegraphics[width=\textwidth]{limi}     
      \end{center}
    \end{column}
  \end{columns}
\end{frame}


\begin{frame}{Haltes migratoires et hivernage importants pour la conservation}
  \begin{columns}
    \begin{column}[c]{0.60\textwidth}
      \begin{itemize}[<+->]
      \item espèces migratrices 
      \item qualités nutritive des haltes migratoires {\tiny
          \cite{Morrison2007,Studds2017}} et de l'hivernage {\tiny
          \cite{Piersma1993,Tulp2009}}  affectent la qualité de la
        reproduction
      \item les ressources peuvent être limitées sur certains sites
        (tropicaux) 
      \item cette limitation entraine de la compétition qui est le
        principale moteur de mortalité des limicoles en hiver {\tiny \cite{Baker1973}}
      \end{itemize}
    \end{column}
    \begin{column}[c]{0.4\textwidth}
      \begin{center}
         \includegraphics[width=\textwidth]{feeding}     
      \end{center}
    \end{column}
  \end{columns}
\end{frame}




\begin{frame}{Interactions}
  \begin{columns}
    \begin{column}[c]{0.6\textwidth}
      \begin{itemize}
      \item comportement territorial sont observés sur les site
        d'hivernage
      \item surtout chez espèces qui chasse à la vue \\ {\footnotesize
          pluvier, chevalier et bécasseau} {\tiny \cite{Colwell2000}}
      \end{itemize}
    \end{column}
    \begin{column}[c]{0.4\textwidth}
       \includegraphics[width=\textwidth]{fight}     
    \end{column}
  \end{columns}
\end{frame}



\begin{frame}{Le chevalier errant (\textit{Tringa incana}, Gmelin, 1789) }
  \begin{columns}
    \begin{column}[c]{0.6\textwidth}
      \begin{itemize}
      \item petit limicole d'env. 110 g
      \item reproduction dans les vallées montagnardes d'amérique du
        nord {\tiny \cite{Gill2015}}
      \item largement distribué dans la zone pacifique pour son
        hivernage {\tiny \cite{GillJr.2002}}
      \item semble territorial même en hiver {\tiny
          \cite{Beichle2001}}\\
        {\footnotesize 2 oiseaux se partagent 700 m de côte (350 m chacun)}
        
      \end{itemize}
    \end{column}
    \begin{column}[c]{0.4\textwidth}
       \includegraphics[width=\textwidth]{KiviKuaka_3_RL_20220306_155702_RL3_4503}     
    \end{column}
  \end{columns}
\end{frame}


\begin{frame}{Un cas original : Tikei}
  \begin{columns}
    \begin{column}[c]{0.55\textwidth}
      \begin{itemize}[<+->]
      \item beaucoup d'études sur les vasières \\ {\footnotesize Europe, Asie, Amérique
        du sud, Australie, Nouvelle-Zélande}
      \item Tikei est une petite ile de Polynésie française
      \item Population de Chevalier errant \\ {\footnotesize
          Env. 25 individus}
        
      \end{itemize}
    \end{column}
    \begin{column}[c]{0.45\textwidth}
      \includegraphics<1>[width=\textwidth]{KiviKuaka_3_RL_20140912_071004_P9120201}
      \includegraphics<2>[width=\textwidth]{KiviKuaka_3_RL_20210127_124920_P1270001}
      \includegraphics<3>[width=0.8\textwidth]{KiviKuaka_3_RL_20220303_153324_RL3_4470}
    \end{column}
  \end{columns}
\end{frame}



\begin{frame}{Hypothèses}
  une espèce territoriale mais une densité importante \\ (env. 25
  chevalier sur l'ile)
 
           \begin{itemize}
 \item Recouvrement des domaines vitaux mais peu des coeurs d'aires
 \item Distances entre oiseaux devraient rester grandes même
   pour les individus proches. 
 \end{itemize}

  
  

\end{frame}




% ===============================================
\section{Materiel et méthodes}
% ===============================================





\begin{frame}{Tikei une petite ile isolée}
  \begin{center}
     \includegraphics<1>[width=0.6\textwidth]{polynesie_sites_1}   
     \includegraphics<2-4>[width=0.6\textwidth]{Tikei_3}
     \includegraphics<5->[width=0.6\textwidth]{KiviKuaka_3_RL_20210128_032720_DJI_0559-1} 
  \end{center}
  \begin{itemize}[<+->]
  \item à $70 km$ de la plus proche ile
  \item $3.9 km$ de long sur $1.6 km$ de large
  \item $4.5 km^2$  
  \item $9 km$ de côte
  \item altitude max $3 m$
  \end{itemize}
\end{frame}

\begin{frame}{Tikei une petite ile isolée}
  \begin{center}
     \includegraphics[width=0.6\textwidth]{PLEO0483}       
  \end{center}
  \begin{itemize}
  \item forêt : $3.2 km^2$
  \item brousaille : $0.3 km^2$
  \item plage : $0.3 km^2$ 
  \item paltier : $0.7 km^2$
  \end{itemize}
\end{frame}



\begin{frame}{Les balises : la contrainte du poids}
  \begin{columns}
    \begin{column}[c]{0.55\textwidth}
      \begin{itemize}
      \item effet des balises sur la survie et la productivité dès
        $1\%$ du poids de l'oiseau {\tiny \cite{Bodey2018}}
      \item règle éthique : pas plus de $5\%$ \\au USA $3\%$ {\tiny \cite{Bridge2011}}
      \end{itemize}
    \end{column}
    \begin{column}[c]{0.45\textwidth}
          \includegraphics[width=\textwidth]{Bridge_et_al_2011_logger}       
    \end{column}
  \end{columns}
\end{frame}


\begin{frame}{Les balises Icarus}
  \begin{columns}
    \begin{column}[c]{0.5\textwidth}
      \begin{itemize}[<+->]
      \item initiative du Max Plank {\tiny \cite{Jetz2022}} 
      \item petite balise GPS de 5g
      \item communique via ISS par VHF
        \begin{itemize}
        \item basse altitude
        \item contrainte sur les aquisitions
        \end{itemize}
      \item objectif de réseau de surveillance {\tiny \cite{Jetz2022}}
      \end{itemize}
         \end{column}
    \begin{column}[c]{0.5\textwidth}
      \begin{center}
        \includegraphics[width=0.5\textwidth]{icarus}
        \vspace{12pt}
        \includegraphics<3-5>[width=\textwidth]{Icarus_AutoE}
        \includegraphics<6->[width=\textwidth]{jetz_et_al_2022_icarus}
       \end{center}
     \end{column}
  \end{columns}
 \end{frame}


 \begin{frame} {Déployment des balises}
   \begin{itemize}
   \item  17 chevaliers ont été équipés du 27 au 28 janvier 2021
   \end{itemize}
   \begin{center}
     \includegraphics<1>[width=.5\textwidth]{KiviKuaka_3_RL_20220307_062331_RL3_4511}
     \includegraphics<2>[width=.5\textwidth]{KiviKuaka_3_RL_20210128_082356_RL3_2007}
   \end{center}
\end{frame}


 \begin{frame} {Déployment des balises}
   \begin{itemize}
   \item  17 chevaliers ont été équipés du 27 au 28 janvier 2021
   \item  13 balises ont envoyées des données
   \end{itemize}
   \begin{center}
     \includegraphics<1>[width=0.8\textwidth]{Chevalier-errant-Tikei-Polynesie-francaise-scaled-e1618956657967-1536x1021}
     \includegraphics<2>[width=0.8\textwidth]{tikei_tattler}
   \end{center}
\end{frame}

\begin{frame}{L'utilisation des habitats}
   \begin{columns}
    \begin{column}[c]{0.4\textwidth}
      à l'exception de T06 les chevaliers n'utilisent pas la forêt et utilise que :
      \begin{itemize}
      \item le platier
      \item la plage
      \item les broussailles
      \end{itemize}
    \end{column}
    \begin{column}[c]{0.6\textwidth}
      \begin{center}
        \includegraphics[width=\textwidth]{gg_bird_habitat-1}
      \end{center}
    \end{column}
  \end{columns}
\end{frame}


\begin{frame}{L'aquisition des données}
  \begin{columns}
    \begin{column}[c]{0.4\textwidth}
      Une aquisition irrégulière et qui s'arrète assez vite :-/
    \end{column}
    \begin{column}[c]{0.6\textwidth}
      \begin{center}
        \includegraphics[width=\textwidth]{Chevalier_errant_loc_per_day}
      \end{center}
    \end{column}
  \end{columns}
\end{frame}


\begin{frame}{L'aquisition des données}
  Une aquisition irrégulière et qui s'arrète assez vite :-/
  \begin{center}
    \includegraphics[width=0.8\textwidth]{tikei_tattler_week}    
  \end{center}
\end{frame}


\begin{frame}{L'aquisition des données}
  \begin{columns}
    \begin{column}[c]{0.4\textwidth}
      Une aquisition qui semble dépendre des passages de l'ISS
    \end{column}
    \begin{column}[c]{0.6\textwidth}
      \begin{center}
        \includegraphics[width=\textwidth]{Tringa_incana_loc_per_day}
      \end{center}
    \end{column}
  \end{columns}
\end{frame}


\begin{frame}{L'aquisition des données}
  Mais des données relativement synchrones :-)\\
  \begin{center}
    \includegraphics[width=\textwidth]{tikei_tattler_nb_data_week_day_hour}
  \end{center}
\end{frame}

\begin{frame}{L'aquisition des données}
  Mais des données relativement synchrones :-)\\
  \begin{center}
    \includegraphics[width=\textwidth]{tikei_tattler_nb_data_week_day_hour}
  \end{center}
    {\footnotesize 4647 couple de données séparé de moins de 15 minutes}
\end{frame}


\begin{frame} {Domainee vitaux}
  \begin{itemize}
  \item Données de localisation sont adaptées à l'estimation des
    domaines vitaux (ou zones utilisées par les indivudus) {\tiny \cite{Kie2010}}
  \item on les calcule avec le package adehabitat {\tiny \cite{Calenge2015,Calenge2006} }
  \end{itemize}
\end{frame}


\begin{frame}{Kernel paramétrage : h ?}
  \begin{itemize}
  \item La fonction de kernel \textit{kernelUD} a un paramètre d'ajustement
    $h$. Par defaut sa valeur \textit{h-ref} est adapté au jeux de
    données qui ont beaucoup de localisations distribuées de manière unimodale. Mais dans le cas
    d'échantillonnage petit ce paramètre va surestimé la taille du domaine
    vital {\tiny \cite{Schuler2014}}.
  \item dans les cas de faible échantillonnage il est suggéré de
    réaliser une recherche d'un \textit{h ad hoc } {\tiny \cite{Schuler2014}}
  \end{itemize}
\end{frame}


\begin{frame}{Kernel paramétrage : h ?}
  Algorithme simple
  \begin{itemize}
  \item pour chaque oiseau
  \item recherche de l'\textit{extent} et du plus petit $h$ pour lequel l'isoligne
    $95\%$ forme un seul polygon simple
  \item algorithme de recherche est un processus simple  essai / erreur
  \end{itemize}
\end{frame}


\begin{frame}{Kernel paramétrage : h ?}
  \begin{center}
     \includegraphics<1>[width=\textwidth]{get_h_accumulation_all}
     \includegraphics<2>[width=\textwidth]{get_h_kernel_T09_red}
     \includegraphics<3>[width=\textwidth]{get_h_kernel_T19_red}
     \includegraphics<4>[width=\textwidth]{get_h_kernel_T22_red}
  \end{center}
\end{frame}


\begin{frame}{Domaine vital et coeur d'air}
  \begin{itemize}[<+->]
  \item   L'isoline $95\%$ représente le domaine vital des oiseaux 
  \item L'isoline $50\%$ correspond au couer de leur domaine vital
    c'est à dire où ils passent plus de la moitier de leur temps
    \tiny{\cite{Benhamou2013,Jourdan2021}} 
  \item intersections avec les habitats utilisés
    \begin{itemize}
    \item le platier
    \item la plage
    \item les brousaille
    \end{itemize}
  \end{itemize}
\end{frame}


\begin{frame}{Domaines vitaux et coeur d'air}
  \begin{center}
     \includegraphics<1>[width=\textwidth]{tikei_tattler_kernel_T09_red}
     \includegraphics<2>[width=\textwidth]{tikei_tattler_kernel_T19_red}
     \includegraphics<3>[width=\textwidth]{tikei_tattler_kernel_T22_red}
  \end{center}
\end{frame}



% ===============================================
\section{Résultats}
% ===============================================

\begin{frame}{Domaine vital et coeur d'air}
  \begin{center}
     \includegraphics<1>[width=0.7\textwidth]{kernel_95_all}
     \includegraphics<2>[width=\textwidth]{intersect_95_50}
   \end{center}
\end{frame}


\begin{frame}{Distances entre individus}
  \begin{center}
     \includegraphics<1>[width=0.8\textwidth]{dis_violin_jour_nuit_kernel}
     \includegraphics<2>[width=0.8\textwidth]{dis_time_jour_nuit_kernel}
      \includegraphics<3>[width=0.9\textwidth]{glm}
   \end{center}
\end{frame}


% ===============================================
\section{Conclusion}
% ===============================================

\begin{frame}{Conclusions}
  \begin{itemize}
  \item les oiseaux semblent territoriaux
  \item les intéraction semblent limitées malgré le recouvrement des
    domaines vitaux
  \end{itemize}
\end{frame}




\begin{frame}[plain]
  \begin{center}
    \includegraphics[width=0.8\textwidth]{KiviKuaka_3_RL_20210209_172324_WhatsApp} \\
   % \vspace{12pt}
    \begin{Large}
      Merci à l'équipe Kivikuaka   
    \end{Large}
  \end{center}
\end{frame}


\begin{frame}[allowframebreaks]
  \begin{tiny}
    \frametitle{Réferences}
    \bibliographystyle{apalike}
    \bibliography{bib_files/biblio}
  \end{tiny}
\end{frame}




\end{document}
